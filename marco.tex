\chapter{Marco Referencial}
\label{sec:marco}

\section{Marco de Antecedentes}
Los juzgados en Colombia est\'an obligados a publicar la informaci\'on de 
los estados judiciales a primera hora h\'abil del d\'ia siguiente al que 
fueron generados, para cumplir con esta exigencia la gran mayor\'ia de 
juzgados instala carteleras en las cuales pegan una o varias hojas 
donde se encuentra impresa la lista de procesos que se notifican por 
estado ese d\'ia. Estas listas son de car\'acter p\'ublico y cualquier persona
que est\'e interesada en ellas puede consultarlas libremente.
\paragraph{}
Para los abogados es importante contar con la informaci\'on de sus 
procesos de manera oportuna y para lograrlo cada uno utiliza el m\'etodo 
que considere m\'as conveniente, algunos visitan los juzgados diariamente,
otros contratan personas que se encargan de revisar los listados y 
recientemente ayudados por las tecnolog\'ias de la informaci\'on han surgido
servicios donde se satisfacen estas necesidades. Las personas encargadas
de la prestaci\'on de este servicio basado en las tecnolog\'ias de la 
informaci\'on, realizan recorridos diarios tomando fotograf\'ias o escaneando
las listas de estados en cada juzgado, para posteriormente procesar la 
informaci\'on y enviarla a sus clientes.
\paragraph{}
En la tesis de maestr\'ia de Fernando Casanovas Mart\'in, llamada 
''Approximate string matching algorithms in art media archives''\footnote{CASANOVAS MART\'IN, Fernando. Approximate string matching algorithms in art media archives.
Tesis de maestr\'ia. Cracovia:  AGH University of Science and Technology.
Faculty of Electrical Engineering, Automatics, Computer Science and Electronics, 2009.} se 
intenta crear un sistema para centralizar y facilitar el acceso de los 
usuarios a un n\'umero significativo de archivos multimedia en Europa. 
Aqu\'i abordan el problema de buscar en los archivos los nombres de los 
autores, donde el sistema desarrollado debe encontrar las coincidencias
tolerando errores de escritura y abreviaciones.
\paragraph{}
El anterior documento es importante porque en el desarrollo de la 
investigaci\'on se eval\'uan distintos algoritmos de coincidencia aproximada
de cadenas, dando un punto de partida importante para el desarrollo de 
la investigaci\'on que se piensa llevar a cabo.
\paragraph{}
A pesar de que existen diversas herramientas que permiten la identificaci\'on
de texto contenido en im\'agenes, el proceso de revisi\'on de estados es 
una actividad que se realiza ‘manualmente’, por lo tanto, es necesario 
desarrollar una herramienta que ayude a agilizar este proceso.

\section{Marco Te\'orico}
\subsection{Coincidencia aproximada de cadenas}
La coincidencia aproximada de cadenas o tambi\'en conocido como b\'usqueda 
difusa de cadenas, contempla encontrar patrones de texto dentro de un 
texto m\'as grande, permitiendo alguna cantidad de errores en la coincidencia.
Esto significa que para que una cadena coincida con otra, no es necesario
que tengan la misma longitud ni que coincidan exactamente todos los 
caracteres.
\paragraph{}
Para poder buscar estas cadenas difusas, 
''lo primero que se necesita es crear un modelo de error, el cual define qu\'e tan diferentes son dos cadenas. Esta idea de qu\'e tan diferentes son dos cadenas es llamado distancia entre cadenas.''\footnote{CASANOVAS MART\'IN, Fernando. Approximate string matching algorithms in art media archives.
Tesis de maestr\'ia. Cracovia:  AGH University of Science and Technology.
Faculty of Electrical Engineering, Automatics, Computer Science and Electronics, 2009. p. 7}

\subsection{Edit Distance}
Edit distance es una distancia que mide la diferencia entre dos 
secuencias o cadenas de texto sobre un alfabeto. Esta distancia mide el
m\'inimo n\'umero de operaciones necesarias para convertir una cadena en la
otra. Una operaci\'on se considera la inserci\'on, la sustituci\'on o la 
remoci\'on de un solo caracter.

\section{Marco Conceptual}
\subsection{Proceso judicial}
Es el mecanismo encargado de proteger el derecho al debido proceso 
establecido por la constituci\'on, donde se garantiza un juez o un 
tribunal para que se encargue de juzgar las actuaciones judiciales y 
administrativas bajo el amparo de la ley. Quien sea sindicado tiene 
derecho a la defensa, a ser representado por un abogado, a presentar 
pruebas y controvertir las decisiones que se tomen en su contra.
\footnote{Congreso de la Rep\'ublica, Constituci\'on Pol\'itica de Colombia (Art\'iculo 29. C\'odigo General del Proceso), Colombia 1991.}


\subsection{Notificaci\'on por estado} 
Notificaci\'on de la decisi\'on de un juez que seg\'un lo establecido en la 
ley no deba hacerse personalmente, esta se hace por medio de anotaci\'on 
en estados los cuales deben ser publicados a la primera hora laboral 
del d\'ia siguiente a la fecha en que es generado.

\subsection{Auto}
Son todas las decisiones de un juez, cualquiera fuere la instancia en 
que se pronuncien, que no sean decisiones sobre las pretensiones de
la demanda o las excepciones que no tengan el car\'acter de previas.

\subsection{Tr\'amite} Si el recurso se formula por escrito, este se mantendr\'a en la secretar\'ia por dos d\'ias en traslado a la parte contraria, sin necesidad de que el juez, lo ordene; surtido el traslado se decidir\'a el recurso. El secretario dar\'a cumplimiento al art\'iculo.
La reposici\'on interpuesta en audiencia y diligencia se decidir\'a all\'i mismo, una vez o\'ida la parte contraria si estuviere presente. Para este fin cada parte podr\'a hacer uso de la palabra hasta por quince minutos. \footnote{Presidencia de la Rep\'ublica de Colombia, Diario Oficial No. 33.150, C\'odigo del Procedimiento Civil.  Colombia, Septiembre de 1970}
\begin{description}
\item[Procesos con tr\'amite:] Proceso que se encuentra abierto y en 
determinado periodo se ha desarrollado acorde lo que  se define como 
tr\'amite en el c\'odigo del procedimiento civil.

\item[Procesos sin tr\'amite:] Proceso que se encuentra abierto pero en 
determinado periodo no se ha desarrollado acorde a lo que se define 
como tr\'amite en el c\'odigo del procedimiento civil.
\end{description}

\subsection{Coincidencia Aproximada de cadenas}  
''La coincidencia aproximada de cadenas es una coincidencia de cadenas 
que permite errores. El objetivo es realizar una coincidencia de 
cadenas en un texto donde una o ambas fuentes han sufrido alg\'un tipo 
de corrupci\'on.''\footnote{CASANOVAS MART\'IN, Fernando. Approximate string matching algorithms in art media archives.
Tesis de maestr\'ia. Cracovia:  AGH University of Science and Technology.
Faculty of Electrical Engineering, Automatics, Computer Science and Electronics, 2009. p. 73}

\subsection{Reconocimiento de texto} 
''El reconocimiento de texto es ejecutado para convertir una imagen 
que contiene texto impreso o manuscrito en un formato que puede ser 
entendido por un computador (Ej. ASCII o unicode)''\footnote{MARINAI, Simone. Introduction to Document Analysis and Recognition. 
\underline{En}: Machine Learning in Document Analysis and Recognition. Florencia: Springer. 2008. p. 8}

\subsection{OCR}
''Los sistemas de reconocimiento de texto frecuentemente dividen las 
palabras en caracteres y luego asignan una clase a cada objeto aislado.
Cuando se trata de texto impreso, esta aproximaci\'on es generalmente 
conocido como Reconocimiento \'Optico de Caracteres (OCR por sus siglas 
en ingl\'es)''\footnote{MARINAI, Simone. Introduction to Document Analysis and Recognition. 
\underline{En}: Machine Learning in Document Analysis and Recognition. Florencia: Springer. 2008. p. 8}


\subsection{Probabilidad}
La probabilidad permite expresar num\'ericamente la posibilidad que un 
evento ocurra.  Se afirma que algo va o no va a ocurrir, pero no se 
est\'a seguro de ello. La probabilidad es el grado de confianza en la 
validez de dicha afirmaci\'on. \footnote{ WALPOLE, Ronald, et al. Probabilidad y estad\'istica para ingenier\'ia y ciencias. Octava Edici\'on. Traductores Javier Enr\'iquez Brito y Victoria Augusta Flores Flores. M\'exico: Pearson Educaci\'on,  2007.}

\pagebreak