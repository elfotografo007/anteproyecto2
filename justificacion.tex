\chapter{Justificaci\'on}
\label{sec:justificacion}
Seg\'un la sociolog\'ia jur\'idica, rama encargada del estudio de la relaci\'on
entre la justicia y la sociedad, los procesos judiciales se generan por
interacciones sociales, ya que estas relaciones est\'an vinculadas a temas
como la violencia pol\'itica, la democracia, el narcotr\'afico, los derechos
humanos y efectos del desarrollo de la econom\'ia como lo es la migraci\'on
del campo a la ciudad. \footnote{CARVAJAL, Jorge. La sociolog\'ia jur\'idica y el derecho. \underline{En}: Revista Proleg\'omenos. Derechos y Valores. Bogot\'a D.C. Julio-diciembre 2011, vol. 28. p. 109-119} 
Es Colombia un pa\'is considerado una econom\'ia en desarrollo, donde se presentan
avances significativos en la econom\'ia con crecimiento del Producto 
Interno Bruto de 4.0\% para el 2010, 6.6\% para el 2011 y 4.0\% para el 2012. 
\footnote{DANE. Cuentas Nacionales Trimestrales - Producto Interno Bruto, Cuarto trimestre 2012. Bogot\'a D.C. 21 de marzo, 2013. [online]  Disponible en internet 
\textless  http://www.dane.gov.co/files/investigaciones/pib/cp\_PIB\_IVtrim12.pdf\textgreater} 
Con un \'indice desarrollo humano calificado como alto, indicador medido 
por las naciones unidas que tiene en cuenta la longevidad, nivel de 
educaci\'on y el nivel de vida alcanzado por la sociedad, valorado en 
0.719 para el 2012, donde 0 es para el nivel de vida m\'as bajo y 1 es 
para el nivel de vida m\'as alto,\footnote{NACIONES UNIDAS. Informe Sobre Desarrollo Humano 2013. El Ascenso Del Sur. Nueva York. 2013.}  pero tambi\'en es un pa\'is con
un alto \'indice de desigualdad social como lo muestra el \'indice gini, 
indicador medido por el Banco Mundial, donde se representa en niveles 
cercanos a cero las econom\'ias con mayor igualdad en la repartici\'on de 
la riqueza y con niveles cercanos a uno las econom\'ias con mayor nivel 
de concentraci\'on de la riqueza, para Colombia el \'indice gini en el 2010 
fue 56.7 y en el 2011 fue de 55.9, \footnote{BANCO MUNDIAL. Indicadores de Desarrollo Mundial. 2013 [Online] Disponible en internet 
\textless  http://datos.bancomundial.org/pais/colombia\textgreater} ubic\'andolo como 
una de las econom\'ias mas desiguales del continente. Tambi\'en es un pa\'is 
donde persiste un conflicto armado que hasta el momento ha dejado m\'as 
de 4,7 millones de desplazados de manera violenta,
\footnote{CENTRO DE MEMORIA HIST\'ORICA. Informe Basta ya Colombia, Memorias de Guerra y Dignidad. 2012. [Online] Disponible en internet 
\textless http://www.centrodememoriahistorica.gov.co/micrositios/informeGeneral\textgreater} 
realidades sociales que bas\'andose en los principios de la sociolog\'ia 
jur\'idica ratifican la tendencia en el incremento de los procesos 
judiciales que se inician en el pa\'is.
\paragraph{}
En un servicio integral de gesti\'on de procesos judiciales es importante
contar con informaci\'on oportuna y a tiempo. Actualmente los juzgados 
publican la informaci\'on en carteleras llamadas notificaci\'on por estado,
las personas a cargo del servicio de gesti\'on de procesos judiciales se 
debe desplazar diariamente a consultar estas carteleras, tomando 
fotograf\'ias que posteriormente se procesan manualmente para integrar 
estos datos a la plataforma. Sumando el factor de crecimiento del 
volumen de procesos en la rama judicial, con un indicador interno 
empresarial de crecimiento de clientes que va ligado al n\'umero de 
procesos registrado en el servicio, la tarea de revisi\'on y comparaci\'on 
de la informaci\'on generada en los juzgados con la registrada en el 
servicio se convierte en algo colosal.
\pagebreak